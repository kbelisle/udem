\documentclass[a4paper,12pt,french]{article}
\RequirePackage[l2tabu, orthodox]{nag}
\usepackage{setspace}
\usepackage[margin=1in]{geometry}
\usepackage{amsmath}
\usepackage{amssymb}

%Enable fr support
\usepackage[utf8]{inputenc}
\usepackage[T1]{fontenc}
\usepackage{lmodern} % load a font with all the characters
\usepackage{babel}
\usepackage{courier}
\usepackage{listings}

%Assign document variables
\date{\today}
\author{Kevin Belisle \& Simon Bernier St-Pierre}
\title{TP2}
\newcommand{\Teacher}{Marc Feeley}
\newcommand{\ClassNum}{IFT-2035}
\newcommand{\ClassName}{Concepts des langages de programmation}
\newcommand{\DateMMMMYYYY}{Décembre 2015}
\newcommand{\Author}{Kevin Belisle}
\newcommand{\Authorr}{Simon Bernier St-Pierre}
\makeatletter

%Custom Header & Footer
\usepackage{fancyhdr}
\pagestyle{fancy}
\fancyhf{}
\fancyhead[L]{\@title}
\fancyhead[R]{\thepage}
\fancyfoot[L]{Kevin Belisle \& Simon Bernier St-Pierre}
\fancyfoot[R]{\DateMMMMYYYY}
\renewcommand{\footrulewidth}{0.4pt}% default is 0pt

% code listings
\lstset{language=C}
\lstset{basicstyle=\footnotesize\ttfamily}

\begin{document}
	\begin{titlepage}
		\begin{center}
			\textsc{\normalsize Université de Montréal}\\[2.5cm]
			
			\textsc{\LARGE \@title}\\[2.5cm]
			
			\textsc{\small Par}\\[0.25cm]
			\textsc{\LARGE \Author}\\[0.25cm]
			\textsc{\normalsize (20018469)}\\[0.25cm]
			\textsc{\LARGE \Authorr}\\[0.25cm]
			\textsc{\normalsize (20031840)}\\[2.5cm]
			
			\textsc{\normalsize Baccalauréat en informatique}\\
			\textsc{\normalsize Faculté des arts et des sciences}\\[2.5cm]
			
			\textsc{\small Travail présenté à \Teacher}\\
			\textsc{\small Dans le cadre du cours \ClassNum}\\
			\textsc{\small \ClassName}\\[2.5cm]
			
			\textsc{\normalsize \DateMMMMYYYY}\\[1.5cm]
		\end{center}
	\end{titlepage}
	\section{Fonctionnement général du programme}
	La calculatrice se divise, dans l'ordre, en 4 étapes : attente de l'entrée de l'utilisateur, analyse et validation de la ligne entrée, évalutation de l'expression et finalement, affichage du résultat. Ces états se répéteront jusqu'à temps que l'utilisateur ferme la calculatrice.\\
	
	Une fois la calculatrice initialisée, elle attend que l'utilisateur lui donne une commande à résoudre. Une fois la commande entrée, la calculatrice valide la syntaxe de la commande, analyse la commande et construit un arbre de syntaxe abstrait (additionner,  assigner, créer un nombre, etc.). Si une erreur survient lors de l'étape précédente, la calulatrice affiche le message d'erreur approprié et retourne à son état initial. Sinon, la calculatrice envoie l'arbre de syntaxe abstraite à l'interprèteur qui construira l'expression résultante. Si une erreur survient, la calculatrice affiche un message d'erreur et retourne à son état initial. Si aucune erreur ne survient, la calculatrice évalue l'expression résultante, affiche le résultat et retourne à son état initial.
	\newpage
	\section{Résolution des problèmes}
	\renewcommand{\thesubsection}{(\alph{subsection})}
	\subsection{Comment se fait l'analyse syntaxique d'une expression et comment est réalisé le traitement d'une expression de longueur quelconque}
	La liste de caractères est séparée en liste de mots à l'aide de la fonction \lstinline$tokens$. Cette fonction sépare une liste de caractères sur les espaces. L'analyseur d'expressions est séparé en plusieurs petites fonctions répondant à une interface identique. Les fonctions prennent en paramètre le mot courant, le reste de la liste de mots et une pile contenant les sous-expressions. La fonction \lstinline$dispatch$ choisit, à partir du premier élément de la liste de mots, quelle fonction d'analyse appeler. Le résultat est un arbre de noeuds représentant l'expression donnée. Les noeuds sont représentés par une paire. Le \lstinline$car$ est un symbole représentant le type du noeud, et le \lstinline$cdr$ représente les arguments (données additionelles).\\
	
	\subsection{Comment se fait le calcul de l'expression}
	L'évaluation d'expressions se fait à l'aide de plusieurs petites fonctions en forme itérative ainsi que de continuations. Les fonctions d'évaluation répondent aussi à une interface commune, elle prennent les arguments d'un noeud, la liste des variables, et une continuation qui doit être appelée avec les données produites, soit un nombre et une liste de variables. L'évalutation débute par le noeud \og racine \fg{}. Ses sous noeuds sont évalués en envoyant des continuations qui modifieront le résultat des sous expressions et appèleront la continuation du noeud parent. Lorsque qu'on atteint les feuilles de l'arbre, on a terminé de parcourir l'arbre. Les feuilles appèlent la continuation qui leur a été donnée, et on remonte donc dans l'arbre à l'envers, chaque continuation modifiant le réultat et la liste des variables au besoin.
	\subsection{Comment se fait l'affectation aux variables}
	L'affectation des variables se fait de manière destructive. Si la variable demandé existe, nous retirons la paire existante et nous la remplaçons par la nouvelle paire, sinon nous ajoutons la nouvelle paire. Nous utilisons une approche assoc (c'est-à-dire utilisant equal? pour comaprer les cles) pour rechercher la cle.
	\subsection{Comment se fait l'affichage des résultats et erreurs}
	\subsection{Comment se fait le traitement des erreurs}
	\newpage
	\section{Expérience de développement C vs Scheme}
	\subsection{Kevin}
	En C, le traitement des variables était plus simple puisque nous travaillions avec un tableau indexé et non pas une liste par association qu'il faut parcourir à chaque fois que nous ajoutons, modifions ou accèdons une variable. Aussi, la gestion des erreurs en C est beaucoup plus facile puisque le fait de lever une exception en C arrête l'execution de la fonction courante et quand Scheme, il faut remonter l'exception manuellement.\\
	
	En Scheme, les operations sur les nombres sont plus facile étant donné que les nombres sont des listes de chiffres et la gestion des listes en Scheme est plus facile quand C. Aussi, il n'y a pas de gestion de la mémoire en Scheme. La continuation de Scheme permet un arbre de syntaxe abstraite beaucoup plus simple et efficace à comparer à la quantité astronomique de structure et de fonction en C.\\
	
	Le style de programmation fonctionnel a été bénéfique pour l'arbre de syntaxe abstraite grâce à la continuation et travailler avec des nombres abitrairement grand grâce à la gestion de liste efficace. Cependant, elle a été détrimentale pour la gestion d'erreur puisqu'il fallait remonter possiblement un erreur un nombre arbitrairement grand fois et la gestion des variables puisqu'il faut parcourir la liste par association pour modifier ou accèder une variable.\\
\end{document}
