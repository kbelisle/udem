\documentclass[a4paper,12pt]{article}
\RequirePackage[l2tabu, orthodox]{nag}
\usepackage{setspace}
\usepackage[margin=1in]{geometry}
\usepackage{amsmath}
\usepackage{amssymb}

%Enable fr support
\usepackage[utf8]{inputenc} 
\usepackage[T1]{fontenc} 
\usepackage{lmodern} % load a font with all the characters

%Assign document variables
\date{\today}
\author{Kevin Belisle \& Simon Bernier St-Pierre}
\title{TP1}
\newcommand{\Teacher}{Marc Feeley}
\newcommand{\ClassNum}{IFT2035}
\newcommand{\ClassName}{Concepts des langages de programmation}
\newcommand{\DateMMMMYYYY}{Octobre 2015}
\newcommand{\Author}{Kevin Belisle}
\newcommand{\Authorr}{Simon Bernier St-Pierre}
\makeatletter

%Custom Header & Footer
\usepackage{fancyhdr}
\pagestyle{fancy}
\fancyhf{}
\fancyhead[L]{\@title}
\fancyhead[R]{\thepage}
\fancyfoot[L]{Kevin Belisle \& Simon Bernier St-Pierre}
\fancyfoot[R]{\DateMMMMYYYY}
\renewcommand{\footrulewidth}{0.4pt}% default is 0pt


\begin{document}
\begin{titlepage} 
	\begin{center}
		\textsc{\normalsize Université de Montréal}\\[2.5cm]
		
		\textsc{\LARGE \@title}\\[2.5cm]
		
		\textsc{\small Par}\\[0.25cm]
		\textsc{\LARGE \Author}\\[0.25cm]
		\textsc{\normalsize (20018469)}\\[0.25cm]
		\textsc{\LARGE \Authorr}\\[0.25cm]
		\textsc{\normalsize (Ton Matricule)}\\[2.5cm]
		
		\textsc{\normalsize Baccalauréat en informatique}\\
		\textsc{\normalsize Faculté des arts et des sciences}\\[2.5cm]
		
		\textsc{\small Travail présenté à \Teacher}\\
		\textsc{\small Dans le cadre du cours \ClassNum}\\
		\textsc{\small \ClassName}\\[2.5cm]
		
		\textsc{\normalsize \DateMMMMYYYY}\\[1.5cm]
	\end{center}
\end{titlepage}
\section{CECI EST UNE SECTION }
	\[Sum: S = A \bigoplus B\]
	\[Carry: C_{out} = AB\]
	\begin{tabular}{cc | cc}
		A&B&$S$&$C_{out}$\\
		\hline
		0&0&0&0\\
		0&1&1&0\\
		1&0&1&0\\
		1&1&0&1\\
	\end{tabular}
\section{Arithmétique modulo}
	\subsection{Modulo 4}
		Définition des ensembles:\\
	\[\{...,0,4,8,12,...\}\]
	\[\{...,1,5,9,13,...\}\]
	\[\{...,2,6,10,14,...\}\]
	\[\{...,3,7,11,15,...\}\]
	À partir de cette représentation, on peut noter les quatre ensembles représentatifs, soient $\bar 0, \bar 1, \bar 2, \bar 3$.\linebreak
	\linebreak
	\begin{tabular}{cc|cc}
		$\bar x$&$\bar y$&$\bar x+\bar y$&$\bar x-\bar y$\\
		\hline
		$\bar 2$&$\bar 0$&$\bar 2$&$\bar 2$\\
		$\bar 2$&$\bar 1$&$\bar 3$&$\bar 1$\\
		$\bar 2$&$\bar 2$&$\bar 0$&$\bar 0$\\
		$\bar 2$&$\bar 3$&$\bar 1$&$\bar 3$\\
	\end{tabular}
	\subsection{Modulo 2}
			Définition des ensembles:\\
		\[\{...,0,2,4,6,...\}\]
		\[\{...,1,3,5,7,...\}\]
		À partir de cette représentation, on peut noter les deux ensembles représentatifs, soient $\bar 0, \bar 1$.\linebreak
		\linebreak
		\begin{tabular}{cc|c}
			$\bar x$&$\bar y$&$\bar x+\bar y$\\
			\hline
			$\bar 0$&$\bar 0$&$\bar 0$\\
			$\bar 0$&$\bar 1$&$\bar 1$\\
			$\bar 1$&$\bar 0$&$\bar 1$\\
			$\bar 1$&$\bar 1$&$\bar 0$\\
		\end{tabular}
		\begin{tabular}{cc|c}
			$A$&$B$&$A \oplus B$\\
			\hline
			$0$&$0$&$0$\\
			$0$&$1$&$1$\\
			$1$&$0$&$1$\\
			$1$&$1$&$0$\\
		\end{tabular}\linebreak
		La tableau est très similaire à la table de vérité du ou exclusif.
		\newpage
\section{Preuve par induction}
Étape de Base:\\
	\[1^2 = \frac{(-1)^{1+1}1(1+1)}{2} = \frac{1\bullet 1 (2)}{2} = 1\]	
Étape Inductive:\\
	Supposons que le formule est vrai pour $n$.\\
	Prouvons que c'est aussi vrai pour $n+1$.
	\[1^2 - 2^2 + 3^2 - ... + (-1)^{n+1}n^2 + (-1)^{n+2}(n+1)^2 = \frac{(-1)^{n+1}n(n+1)}{2}  + (-1)^{n+1+1}(n+1)^2 \]
	\[= \frac{(-1)^{n+1}n(n+1)}{2}  + \frac{2(-1)^{n+1+1}(n+1)^2}{2}\]
	\[= \frac{(-1)^{n+1}n(n+1)}{2}  + \frac{2\bullet-1(-1)^{n+1}(n+1)^2}{2}\]
	\[= \frac{(-1)^{n+1}n(n+1)  -2(-1)^{n+1}(n+1)^2}{2}\]
	\[= \frac{(-1)^{n+1}(n+1)(n  -2(n+1))}{2}\]
	\[= \frac{(-1)^{n+1}(n+1)(-n-2)}{2}\]
	\[= \frac{(-1)^{n+1}(n+1) -1 (n+2)}{2}\]
	\[= \frac{(-1)^{n+2}(n+1)(n+2)}{2}\]
	Alors, la formule est vrai $n+1$ et donc, $1^2 - 2^2 + 3^2 - ... + (-1)^{n+1}n^2 = \frac{(-1)^{n+1}n(n+1)}{2}$
\section{Théorème 1}
	Soit
	\[A = \{a, b, c\}\]
	\[X = P(A)\]
	\[x R y \leftrightarrow x \subseteq y \]
	\[ x < y \leftrightarrow x \subset y  \]
	Prouver que $ x < y \leftrightarrow x \ngeq y$\\
	\linebreak
	\[ x < y \Longrightarrow x \subset y\]
	\[ \Longrightarrow \exists a \in y , a \notin x \]
	\[ \Longrightarrow y \not \subseteq x \]
	\[ \Longrightarrow y \not\leq x \]
\newpage

\end{document}
